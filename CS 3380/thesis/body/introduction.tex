\chapter{Introduction}

This project topic was originally posted by NASA Space Apps Challenges. 
Below is the introduction given on the project page [www.spaceappschallenge.org].
\begin{quote}
On clear or partly sunny days, people might look up at the sky and see straight 
lines of what appear to be clouds or white smoke. These lines are not smoke 
or natural clouds; they are contrails produced by aircraft. Contrails form 
because water vapor from jet engine exhaust passes through a cold and humid 
part of the air at high altitudes. Sometimes the jet that created the contrails 
is not visible overhead because winds aloft have blown the vapor trail into 
the observed area after the jet has passed. Naturally occurring high thin 
cirrus clouds do not form straight lines, they are more diffuse and irregular 
in shape than a contrail. Can an app be developed to help a ground observer 
determine the probability that an aircraft made the thin lines 
of white 'clouds' overhead?

Contrails are potentially important sources of global warming. Contrails have 
been estimated to cause a tropospheric warming of 0.2 to 0.3 degree per decade 
by a general circulation model simulation of contrails [Minnis et al. 2004]. 
Contrails reflect solar radiation and absorb and emit thermal infrared radiation. 
They make a radiative forcing that depends on many factors, especially contrail 
optical depth and coverage [Sassen 1997]. For scientists, knowing the contrails 
is an important way to know about climate changes. 

Viewing the image to figure out the contrails is very inefficient, as, in an image, 
there are other clouds data that make detection of contrails difficult. This 
project uses MatLab to program a solution for this problem. MatLab provides 
many built-in functions useful for image processing in general and this project 
in particular. 
\end{quote}

In this report, I will discuss the background of existing projects, my solution, 
and some results. Also, I give conclusions and future work for this topic.
